\documentclass[a4paper,12pt]{article}
\usepackage[lmargin=2.5cm,rmargin=2.5cm,tmargin=1.5cm,bmargin=1.5cm]{geometry}
\usepackage{amsmath}
\usepackage{amsthm}
\linespread{1.5}

\newtheorem{theorem}{Theorem}
\newtheorem{lemma}{Lemma}

\begin{document}

\title{Graph Theory With Sets}
\author{Jackson Gatenby}
\maketitle

\section{Basic Definitions}

A {\bf graph} $G$ is a 2-tuple $(V(G), E(G))$ where $V(G)$ is a finite non-empty
set and $E(G)$ is a set of 2-element subsets of $V(G)$. Elements of $V(G)$ are
{\bf vertices}, and elements of $E(G)$ are {\bf edges}.

Suppose $e = \{u, v\} \in E(G)$. Then, $e$ is {\bf incident} with $u$ and $v$, the
vertices $u$ and $v$ are {\bf incident} with $e$, and $u$ is {\bf adjacent}
to $v$. Distinct edges $e_1 \neq e_2$ are {\bf adjacent} if $e_1 \cap e_2 \neq
\emptyset$ (note that $|e_1 \cap e_2| = 1$ when $e_1$ and $e_2$ are adjacent).

A {\bf directed graph} $G$ is a 2-tuple $(V(G), E(G))$ where $V(G)$ is a finite
non-empty set of elements, called {\bf vertices}, and $E(G)$ is a set of
2-tuples, called {\bf directed edges}, each formed from distinct
elements of $V(G)$.

A {\bf multigraph} $G$ is a 2-tuple $(V(G), E(G))$, where $V(G)$ is a finite
non-empty set of {\bf vertices}, and $E(G)$ is a multiset of 1-element and
2-element subsets of $V(G)$, called {\bf edges}.

Here we consider graphs, also called {\bf simple graphs}, as opposed to
directed graphs and multigraphs.

$|V(G)|$ is the {\bf order} of the graph $G$, and $|E(G)|$ is its {\bf size}.
The {\bf neighbourhood} of a vertex $v \in V(G)$ is
$N_G(v) = \{u \in V(G) | {u,v} \in E(G)\}$.
The {\bf degree} of a vertex $v$ is $d(v) = |N_G(v)|$.
An {\bf isolated} vertex is one where $\forall e \in E(G), v \not\in e$. It is
simple to check that the degree of an isolated vertex is 0.

A {\bf complete graph} of order $p$, $K_p$, is a graph of order $p$ such that
\[ E(K_p) = \{\{u, v\} \,|\, u, v \in V(K_p), u \neq v\}. \]

The {\bf complement} of a graph $G$ of order $p$ is
$G^c = (V(G), E(K_p) \setminus E(G))$. Note that $e \in E(G)$ iff $e \not\in E(G^c)$.

Two graphs $G$ and $H$ are {\bf isomorphic} if there exists a one-to-one
correspondence $f$ from $V(G)$ to $V(H)$ such that $\{u, v\} \in E(G)$ iff
$\{f(u), f(v)\} \in E(H)$.

A graph $G$ is {\bf bipartite} if there exists $A, B \subseteq V(G)$ where
$A \cup B = V(G)$, $A \cap B = \emptyset$, and
$\forall e \in E(G), e \cap A \neq \emptyset \neq e \cap B$.
We call $A, B$ {\bf partite sets} of $G$.

For two disjoint sets $A, B$ of size $m$ and $n$ respectively,
a {\bf complete bipartite graph}, $K_{m,n}$, is a graph defined as
\[ K_{m,n} = (V(K_{m,n}), E(K_{m,n})) = (A \cup B, \{\{u, v\} | u \in A, v \in B\}).\]
It is trivial (though not unnecessary)
to show that a complete bipartite graph is bipartite.

If $G$ is a graph, then a {\bf subgraph} of $G$ is a graph $H$ (write $H \leq G$)
where $V(H) \subseteq V(G)$ and $E(H) \subseteq E(G)$.
For a set $S \subseteq V(G)$, the subgraph {\bf induced} by $S$ is
$\langle S \rangle = (S, \{e \in E(G) | e \subseteq S\})$. A subgraph is
{\bf spanning} if $V(H) = V(G)$. A subgraph is {\bf proper} if $E(H) \neq E(G)$.

\subsection{Basic Results}

\begin{theorem}
If $G$ is a graph, then
\[\sum_{v \in V(G)} d(v) = 2|E(G)|.\]
\end{theorem}

\begin{proof}
\begin{align*}
\sum_{v \in V(G)} d(v)
& = \sum_{v \in V(G)} |\{e \in E(G) | v \in e \}| \\
& = \sum_{v \in V(G)} \sum_{v \in e \in E(G)} 1 \\
& = \sum_{e \in E(G)} \sum_{V(G) \ni v \in e} 1 \\
& = \sum_{e \in E(G)} 2
= 2 |E(G)|.
\end{align*}
\end{proof}

\begin{lemma}
If $G$ is a graph, then $(G^c)^c = G$.
\end{lemma}

\begin{proof}
Let $p$ be the order of $G$. Then,
$V((G^c)^c) = V(G^c) = V(G)$, and
$E((G^c)^c) = E(K_p) \setminus (E(K_p) \setminus E(G)) = E(G)$, so
$(G^c)^c = G$.
\end{proof}

\begin{theorem}
Graphs $G$ and $H$ are isomorphic iff their complements are isomorphic.
\end{theorem}

\begin{proof}
Suppose $G$ and $H$ are isomorphic, with an isomorphism $f : V(G) \to V(H)$.
Then,
\begin{align*}
\{u, v\} \in E(G^c)
& \iff \{u, v\} \not\in E(G) \\
& \iff \{f(u), f(v)\} \not\in E(H) \\
& \iff \{f(u), f(v)\} \in E(H^c),
\end{align*}
so $f$ is an isomorphism of $G^c$ and $H^c$, so the two are isomorphic.

Conversely, if $G^c$ and $H^c$ are isomorphic, then by above and Lemma 1,
$(G^c)^c = G$ and $(H^c)^c = H$ are isomorphic.
\end{proof}

\begin{lemma}
If $G$ is a bipartite graph on partite sets $A, B$, then for any edge
$\{u,v\} \in E(G)$ where $u \in A$, it holds that $v \in B$.
\end{lemma}

\begin{proof}
Let $e = \{u,v\} \in E(G)$ with $u \in A$. Since $A \cap B = \emptyset$, it
follows that $u \not\in B$, so then $e \cap B \subseteq \{v\}$.
Since $e \cap B \neq \emptyset$, it must be that $e \cap B = \{v\}$, so
$v \in B$.
\end{proof}

\section{Walks, Paths, Cycles}

A {\bf walk} of {\bf length} $n$ (where $n \geq 1$)
on a graph $G$ is a finite sequence of vertices
$\{v_0, v_1, v_2, \dots, v_n\}$ from $G$ such that
each pair of consecutive vertices $v_{i-1},v_i$ in the walk are adjacent\footnote{
The common definition of a walk is as a sequence of alternating vertices and
edges, but the inclusion of edges is unnecessary in simple grapahs.}.
A walk is {\bf closed} if $v_0 = v_n$ and {\bf open} if $v_0 \neq v_n$.

A {\bf path} is an open walk where all vertices are distinct.
A {\bf cycle} is a closed walk where all vertices are distinct except the first
and last, where $v_0 = v_n$.

Two vertices $u$ and $v$ are {\bf connected} if $u = v$ or
there exists a walk from $u$ to $v$.

The {\bf distance} $d(u, v)$ between two connected vertices $u, v$ is the length
of the shortest walk (i.e. walk of least length) from $u$ to $v$. The proof of
Lemma 3 below demonstrates that the shortest walk form $u$ to $v$ is actually a
path.

\begin{lemma}
Two distinct vertices are connected if and only if exists a path between them.
\end{lemma}

\begin{proof}
If a path exists between two distinct vertices, then that path is a walk,
and so the vertices are connected.

Suppose two vertices $v_0 \neq v_n$ are connected by an open walk $W = \{v_0, v_1,
\dots, v_n\}$, choosing a walk such that no other walk has shorter length
(this is possible by the well-ordering of the natural numbers). We show that
$W$ is a path. Assume towards contradiction that not all of $v_1, v_2, \dots,
v_{n-1}$ are distinct, then there are $i,j$ with $0 < i < j < n$
such that $v_i = v_j$. Then, $v_i$ and $v_{j+1}$ are adjacent, so
$\{v_0, v_1, \dots, v_i, v_{j+1}, \dots, v_n\}$
is a walk of length $n - j + i < n$, contradicting that $W$ is the shortest
walk from $v_0$ to $v_n$.
\end{proof}

\begin{lemma}
If $G$ is a graph, connectedness forms an equivalence relation on $V(G)$.
\end{lemma}

\begin{proof}
Reflexivity holds by definition.
Suppose vertices $u$ and $v$ are connected by a walk
$W_1 = \{u, u_1, u_2, \dots, u_{n-1}, v\}$, and $v$ and $w$ are connected by a
walk $W_2 = \{v, v_1, \dots, v_{m-1}, w\}$. Then, it is simple to verify that
$W_1' = \{v, u_{n-1}, u_{n-2}, \dots, u_1, u\}$ is a walk from $v$ to $u$,
so symmetry holds, and that
$W = \{u, u_1, \dots, u_{n-1}, v, v_1, \dots, v_{m-1}, w\}$ is a walk from
$u$ to $w$, so transitivity holds.
\end{proof}

A {\bf component} of a graph $G$ is the induced subgraph of an equivalence class
with respect to connectedness. A graph is {\bf connected} if it has a single
component.

\begin{lemma}
A graph is bipartite if and only if all of its components are bipartite.
\end{lemma}

\begin{proof}
Let $G$ be a graph, and let $C_1, C_2, \dots, C_k$ be the components of $G$.

If $G$ is bipartite with partite sets $A$ and $B$, then consider component
$C_i$, and let $A_i = A \cap V(C_i)$, $B_i = B \cap V(C_i)$. We note that
$A_i \cup B_i = V(C_i)$, since every vertex of $C_i$ is in either $A$ or $B$;
we also note $A_i \cap B_i = A \cap B \cap V(C_i) = \emptyset \cap V(C_i) = \emptyset$.

For all $e \in E(C_i)$, we have that $e \in E(G)$, so $e \cap A \neq \emptyset$,
then let $a \in e \cap A$. Then, $a \in V(C_i) \implies a \in A_i \implies
a \in e \cap A_i \implies e \cap A_i \neq \emptyset$. Similarly,
$e \cap B_i \neq \emptyset$, so $C_i$ is bipartite. Since $C_i$ was arbitrary,
every component of $G$ is bipartite.

If every component $C_i$ is bipartite, with partite sets $A_i$ and $B_i$, then
let $A = \bigcup_{i=1}^{k} A_i$, $B = \bigcup_{i=1}^{k} B_i$. Since each vertex
of $G$ is contained in some component, and therefore some $A_i$ or $B_i$, we
see that $A \cup B = V(G)$. Next, for $i \neq j$, since distinct components
have disjoint vertex sets, $A_i \cap B_j \subseteq V(C_i) \cap V(C_j) = \emptyset$,
so $A_i \cap B_j = \emptyset$; and recall that $A_i \cap B_i = \emptyset$.
By distributivity,
$A \cap B = \bigcup_{i=1}^{k} \bigcup_{j=1}^{k} A_i \cap B_j = \emptyset$.

For all $e = \{u, v\} \in E(G)$, then $u$ and $v$ are in the same component
$C_i$, so $e \in E(C_i)$. Then, $e \cap A_i \neq \emptyset \neq e \cap B_i$.
Since $A_i \subseteq A$, $e \cap A \neq \emptyset$; similarly
$e \cap B \neq \emptyset$. So, $A$ and $B$ form partite sets of $G$.
\end{proof}

\begin{theorem}
A graph $G$ is bipartite if and only if it contains no cycle of odd length.
\end{theorem}

\begin{proof}
Let $G$ be a bipartite graph with partitions $A, B$;
we show that $G$ contains no odd-length cycles.
If $G$ has no cycles, we are done. If $G$ has a cycle
$C = \{v_0, v_1, \dots, v_{n-1}, v_n=v_0\}$ of length $n$,
then WLOG let $v_0 \in A$. Then,
by Lemma 2, $v_1 \in B$, and generally for $0 \leq i < n$,
$v_{i} \in A \Rightarrow v_{i+1} \in B$ and
$v_{i} \in B \Rightarrow v_{i+1} \in A$.

By induction, we conclude that $v_i \in A$ for $i$ even, and $v_i \in B$ for
$i$ odd. Since $v_n = v_0 \in A$, we conclude that $n$ is not odd.

Let $G$ be a graph with no odd-length cycles; we show that $G$ is bipartite
by constructing two sets $A, B$ to act as partite sets of $G$ and $B$.
By Lemma 5, we need only consider connected graphs.
WLOG, choose some $v \in V(G)$. Then, recursively define
the set $A$ to consist of $v$ and all vertices adjacent to elements of $B$, and
the set $B$ to consist of all vertices adjacent to elements of $A$. It is easy
to check this is well-defined. Note that, if $u \in A$ and $v \in B$, then there
is a walk between $u$ and $v$ .

Since $G$ is connected, $A \cup B = V(G)$. Consider an edge
$e = \{a, b\} \in E(G)$, it is clear that $a \in A$ or $a \in B$. Then, since
$b$ is adjacent to $a$, we have $b \in B$ or $b \in A$ respectively.
In each case, $e \cap A \neq \emptyset \neq e \cap B$.

It remains to show that $A \cap B = \emptyset$.
Assume (towards contradiction) that $A \cap B$ is non-empty, and let
$v \in A \cap B$. Then, $v$ is in both $A$ and $B$, so there is a closed walk
$W = \{v_0, v_1, v_2, \dots, v_n\}$ of odd length
where there are no shorter walks of odd length. If all of $v_1, v_2,
\dots, v_{n-1}$ are distinct, then $W$ would be a cycle of odd length, a contradiction.
Otherwise, find $0 < i < j < n$ where $v_i = v_j$ and
$v_{i+1}, v_{i+2}, \dots, v_{j-1}$ are all distinct. Note that
$\{v_i, v_{i+1}, \dots, v_j\}$ forms a cycle in $G$, so $j-i$ is even.
Then, construct the walk $W' = \{v_0, v_1, \dots, v_i, v_{j+1}, \dots, v_n\}$,
which is of odd length. But, $W'$ is shorter than $W$, a contradiction.
So, $A \cap B$ is empty.


\end{proof}

\end{document}
