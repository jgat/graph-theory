\documentclass[a4paper,12pt]{article}
\usepackage[lmargin=2.5cm,rmargin=2.5cm,tmargin=1.5cm,bmargin=1.5cm]{geometry}
\usepackage{amsmath}
\usepackage{amsthm}
\linespread{1.5}

\newtheorem{theorem}{Theorem}
\newtheorem{lemma}{Lemma}

\begin{document}

\title{Graph Theory With Sets}
\author{Jackson Gatenby}
\maketitle

\section{Basic Definitions}

A {\bf graph} $G$ is a 2-tuple $(V(G), E(G))$ where $V(G)$ is a finite non-empty
set and $E(G)$ is a set of 2-element subsets of $V(G)$. Elements of $V(G)$ are
{\bf vertices}, and elements of $E(G)$ are {\bf edges}.

Suppose $e = \{u, v\} \in E(G)$. Then, $e$ is {\bf incident} with $u$ and $v$, the
vertices $u$ and $v$ are {\bf incident} with $e$, and $u$ is {\bf adjacent}
to $v$. Distinct edges $e_1 \neq e_2$ are {\bf adjacent} if $e_1 \cap e_2 \neq
\emptyset$ (note that $|e_1 \cap e_2| = 1$ when $e_1$ and $e_2$ are adjacent).

A {\bf directed graph} $G$ is a 2-tuple $(V(G), E(G))$ where $V(G)$ is a finite
non-empty set of elements, called {\bf vertices}, and $E(G)$ is a set of
2-tuples, called {\bf directed edges}, each formed from distinct
elements of $V(G)$.

A {\bf multigraph} $G$ is a 2-tuple $(V(G), E(G))$, where $V(G)$ is a finite
non-empty set of {\bf vertices}, and $E(G)$ is a multiset of 1-element and
2-element subsets of $V(G)$, called {\bf edges}.

Here we consider graphs, also called {\bf simple graphs}, as opposed to
directed graphs and multigraphs.

$|V(G)|$ is the {\bf order} of the graph $G$, and $|E(G)|$ is its {\bf size}.
The {\bf neighbourhood} of a vertex $v \in V(G)$ is
$N_G(v) = \{u \in V(G) | {u,v} \in E(G)\}$.
The {\bf degree} of a vertex $v$ is $d(v) = |N_G(v)|$.

A {\bf complete graph} of order $p$, $K_p$, is a graph of order $p$ such that
\[ E(K_p) = \{\{u, v\} \,|\, u, v \in V(K_p), u \neq v\}. \]

The {\bf complement} of a graph $G$ of order $p$ is
$G^c = (V(G), E(K_p) \setminus E(G))$. Note that $e \in E(G)$ iff $e \not\in E(G^c)$.

Two graphs $G$ and $H$ are {\bf isomorphic} if there exists a one-to-one
correspondence $f$ from $V(G)$ to $V(H)$ such that $\{u, v\} \in E(G)$ iff
$\{f(u), f(v)\} \in E(H)$.

A graph $G$ is {\bf bipartite} if there exists $A, B \subseteq V(G)$ where
$A \cup B = V(G)$, $A \cap B = \emptyset$, and
$\forall e \in E(G), e \cap A \neq \emptyset \neq e \cap B$.
We call $A, B$ {\bf partite sets} of $G$.

For two disjoint sets $A, B$ of size $m$ and $n$ respectively,
a {\bf complete bipartite graph}, $K_{m,n}$, is a graph defined as
\[ K_{m,n} = (V(K_{m,n}), E(K_{m,n})) = (A \cup B, \{\{u, v\} | u \in A, v \in B\}).\]
It is trivial (though not unnecessary)
to show that a complete bipartite graph is bipartite.

\subsection{Basic Results}

\begin{theorem}
If $G$ is a graph, then
\[\sum_{v \in V(G)} d(v) = 2|E(G)|.\]
\end{theorem}

\begin{proof}
\begin{align*}
\sum_{v \in V(G)} d(v)
& = \sum_{v \in V(G)} |\{e \in E(G) | v \in e \}| \\
& = \sum_{v \in V(G)} \sum_{v \in e \in E(G)} 1 \\
& = \sum_{e \in E(G)} \sum_{V(G) \ni v \in e} 1 \\
& = \sum_{e \in E(G)} 2
= 2 |E(G)|.
\end{align*}
\end{proof}

\begin{lemma}
If $G$ is a graph, then $(G^c)^c = G$.
\end{lemma}

\begin{proof}
Let $p$ be the order of $G$. Then,
$V((G^c)^c) = V(G^c) = V(G)$, and
$E((G^c)^c) = E(K_p) \setminus (E(K_p) \setminus E(G)) = E(G)$, so
$(G^c)^c = G$.
\end{proof}

\begin{theorem}
Graphs $G$ and $H$ are isomorphic iff their complements are isomorphic.
\end{theorem}

\begin{proof}
Suppose $G$ and $H$ are isomorphic, with an isomorphism $f : V(G) \to V(H)$.
Then,
\begin{align*}
\{u, v\} \in E(G^c)
& \iff \{u, v\} \not\in E(G) \\
& \iff \{f(u), f(v)\} \not\in E(H) \\
& \iff \{f(u), f(v)\} \in E(H^c),
\end{align*}
so $f$ is an isomorphism of $G^c$ and $H^c$, so the two are isomorphic.

Conversely, if $G^c$ and $H^c$ are isomorphic, then by above and Lemma 1,
$(G^c)^c = G$ and $(H^c)^c = H$ are isomorphic.
\end{proof}

\begin{lemma}
If $G$ is a bipartite graph on partite sets $A, B$, then for any edge
$\{u,v\} \in E(G)$ where $u \in A$, it holds that $v \in B$.
\end{lemma}

\begin{proof}
Let $e = \{u,v\} \in E(G)$ with $u \in A$. Since $A \cap B = \emptyset$, it
follows that $u \not\in B$, so then $e \cap B \subseteq \{v\}$.
Since $e \cap B \neq \emptyset$, it must be that $e \cap B = \{v\}$, so
$v \in B$.
\end{proof}

\section{Walks, Paths, Cycles}

A {\bf walk} of length $n$ (where $n \geq 1$)
on a graph $G$ is a finite sequence of vertices
$\{v_0, v_1, v_2, \dots, v_n\}$ from $G$ such that
each pair of consecutive vertices $v_{i-1},v_i$ in the walk are adjacent.
A walk is closed if $v_0 = v_n$ and open if $v_0 \neq v_n$.

A {\bf path} is an open walk where all vertices are distinct.
A {\bf cycle} is a closed walk where all vertices are distinct except the first
and last, where $v_0 = v_n$.

Two vertices $u$ and $v$ are {\bf connected} if $u = v$, or
there exists a walk from $u$ to $v$.

The {\bf distance} $d(u, v)$ between two connected vertices $u, v$ is the length
of the shortest path from $u$ to $v$. For instance, the distance between two
adjacent vertices is 1.

\begin{lemma}
Two distinct vertices are connected if and only if exists a path between them.
\end{lemma}

\begin{proof} Exercise. \end{proof} % TODO

\begin{lemma}
If $G$ is a graph, connectedness forms an equivalence relation on $V(G)$.
\end{lemma}

\begin{proof} Reflexivity holds by definition.
Symmetry, Transitivity: Exercise. \end{proof} % TODO

A {\bf component} of a graph $G$ is the induced subgraph of an equivalence class
with respect to connectedness. A graph is {\bf connected} if it has a single
component.

\begin{lemma}
A graph is bipartite if and only if all of its components are bipartite.
\end{lemma}

\begin{proof} Exercise. \end{proof} % TODO

\begin{theorem}
A graph $G$ is bipartite if and only if it contains no cycle of odd length.
\end{theorem}

\begin{proof}
Let $G$ be a bipartite graph with partitions $A, B$;
we show that $G$ contains no odd-length cycles.
If $G$ has no cycles, we are done. If $G$ has a cycle
$C = \{v_0, v_1, \dots, v_{n-1}, v_n=v_0\}$ of length $n$
, then WLOG let $v_0 \in A$. Then,
by Lemma 2, $v_1 \in B$, and generally for $0 \leq i < n$,
$v_{i} \in A \Rightarrow v_{i+1} \in B$ and
$v_{i} \in B \Rightarrow v_{i+1} \in A$.

By induction, we conclude that $v_i \in A$ for $i$ even, and $v_i \in B$ for
$i$ odd. Since $v_n = v_0 \in A$, we conclude that $n$ is not odd.

Let $G$ be a graph with no odd-length cycles; we show that $G$ is bipartite
by constructing two sets $A, B$ to act as partite sets of $G$.
By Lemma 5, we need only consider connected graphs.
WLOG, choose some $v \in V(G)$.\footnote{We can choose such
$v$ as $V(G)$ is nonempty and finite by definition.} Then, recursively define
the set $A$ to consist of $v$ and all vertices adjacent to elements of $B$, and
the set $B$ to consist of all vertices adjacent to elements of $A$. It is easy
to check this is well-defined.

Since $G$ is connected, $A \cup B = V(G)$. Consider an edge
$e = \{a, b\} \in E(G)$, it is clear that $a \in A$ or $a \in B$. Then, since
$b$ is adjacent to $a$, we have $b \in B$ or $b \in A$ respectively.
In each case, $e \cap A \neq \emptyset \neq e \cap B$.

Exercise (TODO): Show that $A \cap B = \emptyset$.
\end{proof}

\end{document}
