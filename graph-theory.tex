\documentclass[a4paper,12pt]{article}
\usepackage[lmargin=2.5cm,rmargin=2.5cm,tmargin=1.5cm,bmargin=1.5cm]{geometry}
\usepackage{amsmath}
\usepackage{amsthm}
\linespread{1.5}

\newtheorem{theorem}{Theorem}
\newtheorem{lemma}{Lemma}

\begin{document}

\title{Graph Theory With Sets}
\author{Jackson Gatenby}
\maketitle

\section{Basic Definitions}

A {\bf graph} $G$ is a 2-tuple $(V(G), E(G))$ where $V(G)$ is a finite non-empty
set and $E(G)$ is a set of 2-element subsets of $V(G)$. Elements of $V(G)$ are
{\bf vertices}, and elements of $E(G)$ are {\bf edges}.

Suppose $e = \{u, v\} \in E(G)$. Then, $e$ is {\bf incident} with $u$ and $v$, the
vertices $u$ and $v$ are {\bf incident} with $e$, and $u$ is {\bf adjacent}
to $v$. Distinct edges $e_1 \neq e_2$ are {\bf adjacent} if $e_1 \cap e_2 \neq
\emptyset$, i.e. if they are not disjoint (note that $|e_1 \cap e_2| = 1$ when
$e_1$ and $e_2$ are adjacent).

A {\bf directed graph} $G$ is a 2-tuple $(V(G), E(G))$ where $V(G)$ is a finite
non-empty set of elements, called {\bf vertices}, and $E(G)$ is a set of
2-tuples, called {\bf directed edges}, each formed from elements of $V(G)$.

A {\bf multigraph} $G$ is a 2-tuple $(V(G), E(G))$, where $V(G)$ is a finite
non-empty set of {\bf vertices}, and $E(G)$ is a multiset of 1-element and
2-element subsets of $V(G)$, called {\bf edges}.

Here we consider graphs, also called {\bf simple graphs}, as opposed to
directed graphs and multigraphs.

$|V(G)|$ is the {\bf order} of the graph $G$, and $|E(G)|$ is its {\bf size}.
The {\bf neighbourhood} of a vertex $v \in V(G)$ is
$N_G(v) = \{e \in E(G) | v \in e\}$.
The {\bf degree} of a vertex $v$ is $d(v) = |N_G(v)|$.

A {\bf complete graph} of order $p$, $K_p$, is a graph of order $p$ such that
\[ E(K_p) = \{\{u, v\} \,|\, u, v \in V(K_p), u \neq v\}. \]

The {\bf complement} of a graph $G$ of order $p$ is
$G^c = (V(G), E(K_p) \setminus E(G))$. Note that $e \in E(G)$ iff $e \not\in E(G^c)$.

Two graphs $G$ and $H$ are {\bf isomorphic} if there exists a one-to-one
correspondence $f$ from $V(G)$ to $V(H)$ such that $\{u, v\} \in E(G)$ iff
$\{f(u), f(v)\} \in E(H)$.

\subsection{Basic Results}

\begin{theorem}
If $G$ is a graph, then
\[\sum_{v \in V(G)} d(v) = 2|E(G)|.\]
\end{theorem}

\begin{proof}
\begin{align*}
\sum_{v \in V(G)} d(v)
& = \sum_{v \in V(G)} |\{e \in E(G) | v \in e \}| \\
& = \sum_{v \in V(G)} \sum_{v \in e \in E(G)} 1 \\
& = \sum_{e \in E(G)} \sum_{V(G) \ni v \in e} 1 \\
& = \sum_{e \in E(G)} 2
= 2 |E(G)|.
\end{align*}
\end{proof}

\begin{lemma}
If $G$ is a graph, then $(G^c)^c = G$.
\end{lemma}

\begin{proof}
Let $p$ be the order of $G$. Then,
$V((G^c)^c) = V(G^c) = V(G)$, and
$E((G^c)^c) = E(K_p) \setminus (E(K_p) \setminus E(G)) = E(G)$, so
$(G^c)^c = G$.
\end{proof}

\begin{theorem}
Graphs $G$ and $H$ are isomorphic iff their complements are isomorphic.
\end{theorem}

\begin{proof}
Suppose $G$ and $H$ are isomorphic, with an isomorphism $f : V(G) \to V(H)$.
Then,
\begin{align*}
\{u, v\} \in E(G^c)
& \iff \{u, v\} \not\in E(G) \\
& \iff \{f(u), f(v)\} \not\in E(H) \\
& \iff \{f(u), f(v)\} \in E(H^c),
\end{align*}
so $f$ is an isomorphism of $G^c$ and $H^c$, so the two are isomorphic.

Conversely, if $G^c$ and $H^c$ are isomorphic, then by above and Lemma 1,
$(G^c)^c = G$ and $(H^c)^c = H$ are isomorphic.
\end{proof}


\end{document}
